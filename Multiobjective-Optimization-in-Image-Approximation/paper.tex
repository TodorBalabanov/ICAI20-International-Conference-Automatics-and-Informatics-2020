\documentclass[conference]{IEEEtran}
\IEEEoverridecommandlockouts

\usepackage{cite}
\usepackage{amsmath,amssymb,amsfonts}
\usepackage{algorithmic}
\usepackage{graphicx}
\usepackage{textcomp}
\usepackage{xcolor}
\def\BibTeX{{\rm B\kern-.05em{\sc i\kern-.025em b}\kern-.08em T\kern-.1667em\lower.7ex\hbox{E}\kern-.125emX}}

\begin{document}

\title{Multiobjective Optimization in Image Approximation \\ {}
\thanks{The publication has been prepared with the support of Velbazhd Software LLC and Bulgarian Ministry of Education and Science according to the research project No. {D01–205/23.11.2018}.}}

\author{\IEEEauthorblockN{Plamen Petrov}
\IEEEauthorblockA{\textit{Institute of Information and Communication Technologies} \\
\textit{Bulgarian Academy of Sciences} \\
1113 Sofia, Bulgaria \\
p.petrov@iit.bas.bg}
\and
\IEEEauthorblockN{Georgi Kostadinov}
\IEEEauthorblockA{\textit{Institute of Information and Communication Technologies} \\
\textit{Bulgarian Academy of Sciences} \\
1113 Sofia, Bulgaria \\
g.kostadinov@iit.bas.bg}
\and
\IEEEauthorblockN{Petar Zhivkov}
\IEEEauthorblockA{\textit{Institute of Information and Communication Technologies} \\
\textit{Bulgarian Academy of Sciences} \\
1113 Sofia, Bulgaria \\
City, Country \\
pzhivkov@iit.bas.bg}
\and
\IEEEauthorblockN{Veneta Velichkova}
\IEEEauthorblockA{\textit{Institute of Information and Communication Technologies} \\
\textit{Bulgarian Academy of Sciences} \\
1113 Sofia, Bulgaria \\
vvelichkova@iit.bas.bg}
\and
\IEEEauthorblockN{Stoyan Ivanov}
\IEEEauthorblockA{\textit{Institute of Information and Communication Technologies} \\
\textit{Bulgarian Academy of Sciences} \\
1113 Sofia, Bulgaria \\
et\_idea@abv.bg}
\and
\IEEEauthorblockN{Todor Balabanov}
\IEEEauthorblockA{\textit{Institute of Information and Communication Technologies} \\
\textit{Bulgarian Academy of Sciences} \\
1113 Sofia, Bulgaria \\
0000-0003-3139-069X}
}

%
% Plamen Petrov - p.petrov@iit.bas.bg
% Georgi Kostadinov - g.kostadinov@iit.bas.bg
% Petar Zhivkov - pzhivkov@iit.bas.bg
% Veneta Velichkova - vvelichkova@iit.bas.bg
% Stoyan Ivanov - et_idea@abv.bg
% Todor Balabanov - todorb@iinf.bas.bg
%
% Bulgarian Academy of Sciences
% Institute of Information and Communication Technologies
% acad. Georgi Bonchev Str., block 2, office 514
% 1113 Sofia, Bulgaria
% http://iict.bas.bg/
%

\maketitle

\begin{abstract}
There are two common ways for the representation of images - as pixels (raster graphics) or as a list of geometric primitives (vector graphics). Both ways have their advantages and disadvantages and in some situations conversion between them is needed. Conversion from vector graphics to raster graphics is relatively easy and it is done by a process called rasterization. The opposite conversion (from raster to vector) is much harder and less reliable. The process is called vectorization and in the case of images, it is related to color reduction and information loss. The goal in this research is 16M colors bitmap images to be approximated with 12 colors vectorized form in which the base primitive is an ellipse. The process of vectorization is done with genetic algorithms as one of the most efficient metaheuristic tools for global optimization. The evaluation of the fitness value in the evolution process implemented in the genetic algorithms uses three different objective functions: 1) Average Euclidean distance between the pixels of the original image and the approximated image; 2) The size of the blanks spaces in the approximated image; and 3) The number of graphic primitives used for the approximation. 
\end{abstract}

\begin{IEEEkeywords}
image approximation, genetic algorithms, colors reduction, image vectorization
\end{IEEEkeywords}

\section{Introduction}

Packing problems are well known in geometry from centuries. The goal is to pack certain objects into a container. The most popular case of this problem is when a single container should be filled as dense as possible \cite{Lodi-Martello-Monaci-2002}. There is a dualism for each packing problem as a covering problem. In covering problems it is asked how many objects are required in such organization that will cover the most area in the container. This problem is defined in two cases where object can or can not overlap. The most popular problems in two-dimensional space are: Packing circles in a circle \cite{Fodor-2003}; Packing circles in a square \cite{Huang-Ye-2010}; Packing circles in an isosceles or right triangle \cite{Xu-1996}; Packing circles an equilateral triangle \cite{Nurmela-2000}; Packing squares in a square \cite{Stromquist-2003}.

\section{Conclusion}

\section*{Acknowledgment}

This research is funded by Velbazhd Software LLC and it is partially supported by the Bulgarian Ministry of Education and Science (contract D01–205/23.11.2018) under the National Scientific Program ``Information and Communication Technologies for a Single Digital Market in Science, Education and Security (ICTinSES)'', approved by DCM \# 577/17.08.2018.

\begin{thebibliography}{00}

\bibitem{Lodi-Martello-Monaci-2002} A. Lodi, S. Martello, M. Monaci, ``Two-dimensional packing problems: A survey'', European Journal of Operational Research (Elsevier), vol. 141, no. 2, 2002, pp. 241--252.

\bibitem{Fodor-2003} F. Fodor, ``The Densest Packing of 13 Congruent Circles in a Circle'', Beitrage zur Algebra und Geometrie, Contributions to Algebra and Geometry, vol. 44, no. 2, 2003, pp. 431--440.

\bibitem{Huang-Ye-2010} W. Huang, T. Ye, ``Greedy vacancy search algorithm for packing equal circles in a square'', Operations Research Letters, vol. 38, 2010, pp. 378--382.

\bibitem{Xu-1996} Y. Xu, ``On the minimum distance determined by n ($<$= 7) points in an isoscele right triangle'', Acta Mathematicae Applicatae Sinica, vol. 12, no. 2, 1996, pp. 169--175.

\bibitem{Nurmela-2000} K. Nurmela, ``Conjecturally optimal coverings of an equilateral triangle with up to 36 equal circles'', Experimental Mathematics, vol. 9, no. 2, 2000, pp. 241--250.

\bibitem{Stromquist-2003} W. Stromquist, ``Packing 10 or 11 unit squares in a square'', Electronic Journal of Combinatorics, vol. 10, no. 8, 2003, pp. 1--11.

\end{thebibliography}

\end{document}
